\documentclass{article}

%package setup
\usepackage{graphicx}
\usepackage{amsmath}
\usepackage{fancyhdr}
\usepackage[margin=1in]{geometry}
\usepackage{comment}
\usepackage{placeins}
\usepackage{parskip}
\usepackage{subcaption}
\usepackage{appendix}
\usepackage{soul}
\usepackage{comment}
\PassOptionsToPackage{hyphens}{url}\usepackage[hidelinks]{hyperref}
\usepackage{matlab-prettifier}
\usepackage{minted}
\usepackage{enumitem}
\usepackage{float}
\usepackage{textcomp, gensymb}
\usepackage{tikz}
\usepackage{booktabs}
\usepackage{array}
\usepackage{tabularx}
\usetikzlibrary{arrows.meta, positioning, decorations.pathmorphing}
\usepackage{caption}

\pagestyle{fancy}
\fancyhf{} % Clear header/footer settings
\rhead{\thepage} % Page number on the right in the header
\lhead{ASE162M Report 4} % Your report title on the left

\begin{document}

\begin{titlepage}
  \centering
  \includegraphics[width=10cm]{ase-logo-formal.png}  % Adjust the width as needed
  \vspace{1cm}  % Add some vertical space
 
  \Large \textbf{ASE 162M High-Speed Aerodynamics}\\
  \large \textbf{Section 14275}\\
  \vspace{0.5cm}
  \textbf{Tuesday: 4:00 - 6:00 pm}\\
 
  \vspace{1cm}
 
  \hrule
  \vspace{0.5cm}
 
  \begin{center}
  \Huge \textbf{Final Exam}\\
  \end{center}
 
  \vspace{0.5cm}
  \hrule
 
  \vspace{1cm}  
 
  \normalsize \textbf{Andrew Doty}\\
  \normalsize \textbf{Due Date: December 14, 2024}
 
\end{titlepage}
\newpage

\tableofcontents
\thispagestyle{empty}
\newpage

\section{Paper 1: Visualization of Supersonic Free and Confined Jet using Planar Laser Mie Scattering Technique}

\subsection{Paper Information}
\textbf{Title:} Visualization of Supersonic Free and Confined Jet using Planar Laser Mie Scattering Technique \\
\textbf{Authors:} S.K. Karthick, G. Jagadeesh and K.P.J. Reddy \\
\textbf{Publication:} Journal of the Indian Institute of Science \\
\textbf{Publication Date:} January-March 2016

\subsection{Citation (Chicago Format)}
Karthick, S.K., G. Jagadeesh, and K.P.J. Reddy. "Visualization of Supersonic Free and Confined Jet using Planar Laser Mie Scattering Technique." Journal of the Indian Institute of Science 96, no. 1 (2016): 29-45.

\subsection{Experimental Technique}
The experimental technique used is Planar Laser Mie Scattering (PLMS), which is an optical flow visualization method.

\subsection{Methodology}
PLMS works by illuminating particles in the flow field with a laser sheet and capturing the scattered light. The technique involves:
\begin{itemize}
    \item Using a high-power laser converted into a sheet (thickness ~0.5 mm).
    \item Seeding the flow with particles (DOP or TiO2).
    \item Capturing scattered light using high-speed cameras.
    \item Processing images to extract flow features.
\end{itemize}

\subsection{Experimental Apparatus}
The setup included:
\begin{itemize}
    \item A Mach 2.0 supersonic wind tunnel with optical access.
    \item Nd-YAG laser (532 nm, 500 mJ) for free jet studies.
    \item Nd-YLF laser (527 nm, 24 mJ) for confined jet studies.
    \item Phantom Miro 110 high-speed camera.
    \item An in-house designed seeder unit with a modified Laskin nozzle.
    \item Optical glass windows for flow visualization.
\end{itemize}

\subsection{Flow Application}
The technique was applied to:
\begin{itemize}
    \item Supersonic axisymmetric free jet (Mach 1.37-2.5).
    \item Supersonic rectangular confined jet (Mach 2.0).
\end{itemize}

The authors chose PLMS because it provides:
\begin{itemize}
    \item Instantaneous flow field visualization.
    \item The ability to capture both large-scale structures and fine details.
    \item Non-intrusive measurement capability.
\end{itemize}

\subsection{Key Conclusions}
\begin{enumerate}
    \item PLMS successfully captured important flow features, including the Mach disc, shear layer instability, and shock cells in supersonic jets.
    \item The shock cell spacing increases with increasing Mach number ratio.
    \item Particle lag significantly affects the magnitude of Reynolds stresses in supersonic flow measurements.
\end{enumerate}

\subsection{Suggested Improvements}
\begin{enumerate}
    \item Implement advanced particle tracking algorithms to better quantify particle lag effects.
    \item Develop improved seeding methods for near-wall regions.
    \item Extend the technique to higher Mach numbers with better temperature control to prevent window fogging.
\end{enumerate}

\subsection{Insights Gained}
\begin{enumerate}
    \item Particle selection and seeding methodology are crucial for accurate flow visualization in supersonic flows.
    \item The semi-local scaling shows better agreement with incompressible flows than conventional Morkovin's scaling.
    \item Proper image processing techniques are essential for extracting meaningful flow features from raw PLMS data.
\end{enumerate}


\section{Paper 2: Particle Lag in Supersonic Turbulent Boundary Layers}

\subsection{Paper Information}
\textbf{Title:} The Effect of Particle Lag on Supersonic Turbulent Boundary Layer Statistics \\
\textbf{Authors:} K. Todd Lowe, Gwibo Byun and Roger L. Simpson \\
\textbf{Publication:} AIAA SciTech Forum \\
\textbf{Publication Date:} January 2014

\subsection{Citation (Chicago Format)}
Lowe, K. Todd, Gwibo Byun, and Roger L. Simpson. "The Effect of Particle Lag on Supersonic Turbulent Boundary Layer Statistics." In 52nd Aerospace Sciences Meeting, AIAA SciTech Forum. 2014.

\subsection{Experimental Technique}
The study uses Laser Doppler Velocimetry (LDV) combined with particle lag analysis to study supersonic turbulent boundary layers.

\subsection{Methodology}
The technique involves:
\begin{itemize}
    \item Three-component laser Doppler velocimetry measurements.
    \item Analysis of particle response to turbulent fluctuations.
    \item Application of a linear first-order lag correction scheme.
    \item Use of Reynolds stress spectral functions for corrections.
\end{itemize}

\subsection{Experimental Apparatus}
Key components included:
\begin{itemize}
    \item Mach 2.0 supersonic wind tunnel with optical access.
    \item Small supersonic LDV system with 3 measurement volumes.
    \item Diode-pumped solid state laser (532 nm, 1.5 W).
    \item Argon ion laser (514.5 nm and 488 nm, 1.9 W).
    \item High-quality optical glass window for LDV measurements.
    \item DOP seeding system with modified Laskin nozzle.
\end{itemize}

\subsection{Flow Application}
The technique was applied to:
\begin{itemize}
    \item Mach 2.0 supersonic smooth wall turbulent boundary layer.
    \item Reynolds number based on momentum thickness of approximately 15,582.
\end{itemize}

The authors chose this technique because:
\begin{itemize}
    \item It provides detailed turbulence statistics.
    \item It allows for quantification of particle lag effects.
    \item It enables correction of Reynolds stress measurements.
\end{itemize}

\subsection{Key Conclusions}
\begin{enumerate}
    \item Particle lag significantly affects the magnitude of turbulence quantities in high-speed flows.
    \item Normal-to-wall stress is most susceptible to frequency filtering effects due to particle lag.
    \item Semi-local scaling shows better agreement with incompressible flows than conventional Morkovin's scaling.
\end{enumerate}

\subsection{Suggested Improvements}
\begin{enumerate}
    \item Develop direct measurement techniques for Reynolds stress spectra in supersonic flows.
    \item Implement advanced particle tracking methods for better lag quantification.
    \item Extend the study to higher Mach numbers and different flow configurations.
\end{enumerate}


\section{Conclusion}

\begin{enumerate}
    \item Both studies highlight the critical importance of particle behavior in supersonic flow measurements and the importance of having a good reference mask, accurate particle tracking, and a high-quality camera.
    \item The complementary nature of PLMS and LDV techniques provides comprehensive flow field information when both are used together.
    \item Particle lag effects must be carefully considered in both visualization and quantitative measurements.
\end{enumerate}


\end{document}